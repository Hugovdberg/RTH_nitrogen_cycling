% arara: lualatex
% arara: lualatex
% arara: biber
% arara: lualatex
\documentclass[headings=standardclasses]{scrartcl}

\usepackage{amsmath}
\usepackage{mathtools}
\usepackage{physics}
\usepackage[version=4]{mhchem}
\usepackage{cancel}
\usepackage{siunitx}
\sisetup{per-mode = symbol-or-fraction}
% \usepackage{fontspec}
% \setmainfont{Linux Libertine O}
\usepackage[bold-style=ISO]{unicode-math}
% \setmainfont{XITS}
% \setmathfont{XITS Math}
\setmainfont{Libertinus Serif}
\setmathfont{Libertinus Math}
\usepackage{bbm}
\usepackage[activate={true,nocompatibility}]{microtype}
\usepackage{polyglossia}
\setdefaultlanguage{english}
\usepackage[english=british]{csquotes}

\usepackage[useregional]{datetime2}
\usepackage[shortlabels]{enumitem}

\usepackage[
    style=authoryear-comp,
    dashed=false,
    sorting=nyvt,
    sortlocale=auto,
    sortcites=true,
    maxcitenames=3,
    maxbibnames=100,
    useprefix=false,
    alldates=long,
    language=british,
    clearlang=true,
    % giveninits=true,
    doi=true,
    url=false,
    backref=true,
    abbreviate=false,
    backend=biber]{biblatex}
\addglobalbib{/home/hugo/Documenten/Studie/studie.bib}

\usepackage{booktabs}

\usepackage{listings}
\lstset{
    language=R,
    basicstyle=\footnotesize\ttfamily,
    keywordstyle=\bfseries\color{green!40!black},
    commentstyle=\itshape\color{purple!40!black},
    identifierstyle=\color{blue},
    stringstyle=\color{orange},
    breaklines=true,
}

\usepackage{graphicx}
\usepackage{tikz}
\usetikzlibrary{arrows, arrows.meta, patterns, shapes.geometric, positioning}

\usepackage[%
    bookmarks=true,
    unicode=true,
    pdfusetitle=true,
    hidelinks
]{hyperref}

\title{\texorpdfstring{\vspace{-2cm}Nitrogen Cycling in a River-Aquifer system}{Nitrogen Cycling in a River-Aquifer system}}
\author{Hugo van den Berg \and Max Lamberts \and Timon Smeets}
\date{\DTMtoday\vspace{-0.5cm}}

\renewcommand*{\RNfont}{\scshape}
\newcommand*\mean[1]{\bar{#1}}

% \DeclareMathOperator{\Div}{div}
% \DeclareMathOperator{\Grad}{grad}
% \DeclareMathOperator{\wellf}{W}

\DeclareSIUnit\atmosphere{atm}
\DeclareSIUnit\ppm{ppm}
\DeclareSIUnit\year{yr}

% \definecolor{soil}{rgb}{0.93,0.91,0.41}
% \definecolor{clay}{rgb}{0.2,0.4,0}
% \definecolor{water}{rgb}{0.6,0.8,1}

% \makeatletter
% \pgfdeclarepatternformonly[\hatchdistance,\hatchthickness]{flexible hatch}
% {\pgfqpoint{0pt}{0pt}}
% {\pgfqpoint{\hatchdistance}{\hatchdistance}}
% {\pgfpoint{\hatchdistance-1pt}{\hatchdistance-1pt}}%
% {
%     \pgfsetcolor{\tikz@pattern@color}
%     \pgfsetlinewidth{\hatchthickness}
%     \pgfpathmoveto{\pgfqpoint{0pt}{0pt}}
%     \pgfpathlineto{\pgfqpoint{\hatchdistance}{\hatchdistance}}
%     \pgfusepath{stroke}
% }
% \makeatother

\tikzset{
    box/.style={
        rectangle,
        draw=black!50,
        minimum size=2cm,
        align=center
    }
}

\begin{document}
\maketitle

\section{Introduction}
The aim of this project is to understand the role of organic matter
mineralization in the removal of dissolved organics entering the groundwater
from a contaminated river. Specifically, your task is to write a model that can
predict the concentrations of DOM, \ce{O2}, \ce{NO3}, \ce{NH3} and \ce{N2} along
the flow path of water from the river through the aquifer. First, you will use
the model to fit the given profiles of \ce{O2} and \ce{NH3} in order to obtain
the rate constants of aerobic DOM degradation, nitrification and groundwater
aeration.
Then you will apply the model to predict how the quality of potentially drinking
water extracted from a well in the aquifer depends on the concentrations of DOM,
\ce{NO3} and \ce{NH3} in the river. Your ultimate question is: How does the
coupling of processes of aerobic respiration, nitrification-denitrification
and groundwater aeration determine water quality along an aquifer adjacent to a
river contaminated with organics?

\section{Methods}

To determine the infiltration of nitrogen from a river into the groundwater,
several chemical equations have to be identified. First of all, there is aerobic
respiration of dissolved organic matter (DOM). For the purposes of this model DOM
is simplified to \ce{(CH2O)106(NH3)16}, where the ratio between carbon and
nitrogen is set to the Redfield ratio. Aerobic respiration can then be described
by the following stoichiometric equation:
\begin{equation}
    \ce{(CH2O)106(NH3)16 + 106O2 -> 106CO2 + 106H2O + 16NH3}.\label{eq:stoi_aero}
\end{equation}
When oxygen is depleted, organic matter degradation is also possible through
denitrification. Through this process nitrogen is lost from the system in gaseous
form:
\begin{equation}
    \ce{5(CH2O)106(NH3)16 + 424NO3- + 424H+ -> 530CO2 + 742H2O + 80NH3 + 212N2}.\label{eq:stoi_denit}
\end{equation}
Both these processes lead to the formation of ammonium, which under oxic
conditions can be transformed to nitrate in a process called nitrification:
\begin{equation}
    \ce{NH3 + 2O2 -> NO3- + H2O + H+}.\label{eq:stoi_nit}
\end{equation}

From these equations several species were identified as relevant for the
model. First of all a primary goal of the model is to determine suitability of
the groundwater for drinking water production. The regulations limit the
concentrations of DOM, \ce{NO3}, and \ce{NH3}. Therefore, these parameters have
to be modeled explicitly.
Secondly oxygen is modeled for regulation of the processes, and nitrogen is
modeled to track the final loss of nitrogen from the groundwater.

On top of the chemical processes in the groundwater, also transport from the
river into the aquifer is taken into account. To keep the model relatively
simple

\subsection{Assumptions}
To simplify the problem the following points are assumed:
\begin{itemize}
    \item River and groundwater directly connected, no unsaturated zone in between
    \item Only 1 dimensional transport along streamline
    \item Once nitrogen degases it is permanently lost from the system
    \item DOM is only removed through aerobic respiration and denitrification, other removal processes are ignored.
\end{itemize}

\subsection{Equations}

\begin{align}
    \pdv{[DOM]}{t} &= D\pdv[2]{[DOM]}{x} - u\pdv{[DOM]}{x} - R_{org.rem}\label{eq:diff_dom}\\
    \pdv{[\ce{O2}]}{t} &= D\pdv[2]{[\ce{O2}]}{x} - u\pdv{[\ce{O2}]}{x} - R_{aero} - R_{nit} + R_{aeration}\label{eq:diff_O2}\\
    \pdv{[\ce{NO3}]}{t} &= D\pdv[2]{[\ce{NO3}]}{x} - u\pdv{[\ce{NO3}]}{x} - R_{denit} + R_{nit}\label{eq:diff_NO3}\\
    \pdv{[\ce{NH3}]}{t} &= D\pdv[2]{[\ce{NH3}]}{x} - u\pdv{[\ce{NH3}]}{x} + R_{org.rem} - R_{nit}\label{eq:diff_NH3}\\
    \pdv{[\ce{N2}]}{t} &= D\pdv[2]{[\ce{N2}]}{x} - u\pdv{[\ce{N2}]}{x} + R_{denit} + R_{degas}\label{eq:diff_N2}
\end{align}

\begin{align}
    R_{org.rem} &= k_1 [DOM] = R_{aero} + R_{denit}\label{eq:rate_org_rem}\\
    R_{aero} &= R_{org.rem} \left(\frac{[\ce{O2}]}{k_{\ce{O2}} + [\ce{O2}]}\right)\label{eq:rate_aero}\\
    R_{denit} &= R_{org.rem} \left(\frac{[\ce{NO3}]}{k_{\ce{NO3}} + [\ce{NO3}]}\right) \left(\frac{k_{\ce{O2}}}{k_{\ce{O2}} + [\ce{O2}]}\right)\label{eq:rate_denit}\\
    R_{nit} &= k_2 [\ce{NH3}][\ce{O2}]\label{eq:rate_nit}\\
    R_{aeration} &= k_3 \left(1 - \frac{[\ce{O2}]}{S_{\ce{O2}}}\right)\label{eq:rate_aeration}\\
    R_{degas} &= k_3 \left(1 - \frac{[\ce{N2}]}{S_{\ce{N2}}}\right)\label{eq:rate_degas}
\end{align}

The following values are given as initial guesses for the fitting.
\begin{align}
    k_1 &\approx \SI{1e-3}{\per\hour}\\
    k_2 &\approx \SI{5e-4}{\per\micro\mole\of{\ce{O2}}\per\liter\per\hour}\\
    \frac{k_3}{S_{\ce{O2}}} &\approx \SI{5e-4}{\per\hour}
\end{align}
The value for \(k_3\) is given as a fraction because the notation of the rate law is
different.

From the given values \autoref{tab:parameters} we can calculate the dispersion
coefficient:
\begin{align}
    D &\approx \alpha_L u\\
        &= \SI{10}{\centi\meter\per\hour} \times \SI{1.5}{\meter}\\
        &= \SI{0.15}{\square\meter\per\hour}\\
\end{align}

\begin{table}
    \centering
    \caption{\label{tab:parameters}Given parameters of the system}
    \begin{tabular}{lSs}
        \toprule
        Parameter                           & {Value}   & {Unit}\\
        \midrule
        Velocity, \(u\)                     & 10        & \centi\meter\per\hour\\
        Porosity, \(\phi\)                  & 0.4       & \cubic\meter\of{pw}\per\cubic\meter\\
        Dispersivity, \(\alpha_L\)          & 1.5       & \meter\\
        Affinity \ce{O2}, \(k_{\ce{O2}}\)   & 20        & \micro\mole\per\liter\\
        Affinity \ce{NO3}, \(k_{\ce{NO3}}\) & 35        & \micro\mole\per\liter\\
        Temperature, \(T\)                  & 10        & \celsius\\
        Salinity, \(S\)                     & 0         &\\
        \bottomrule
    \end{tabular}
\end{table}


\subsection{Boundary conditions}
\begin{table}
    \centering
    \caption{\label{tab:boundary_conditions}Boundary conditions for the state variables}
    \begin{tabular}{lll}
        \toprule
        Parameter   & Upper &   Lower\\
        \midrule
        DOM     & \([DOM] = \SI{6}{\milli\gram\of{C}\per\liter} = \SI{0.5}{\milli\mole\of{C}\per\liter} = \SI{4.71}{\micro\mole\of{DOM}\per\liter}\) & \(\pdv{[DOM]}{x} = 0\)\\
        \ce{O2} & \([\ce{O2}] = \SI{210}{\micro\mole\of{O2}\per\liter}\) & \(\pdv{[\ce{O2}]}{x} = 0\)\\
        \ce{NO3} & \([\ce{NO3}] = \SI{100}{\micro\mole\of{NO3}\per\liter}\) & \(\pdv{[\ce{NO3}]}{x} = 0\)\\
        \ce{NH3} & \([\ce{NH3}] = \SI{0}{\micro\mole\of{NO3}\per\liter}\) & \(\pdv{[\ce{NH3}]}{x} = 0\)\\
        \ce{N2} & \([\ce{N2}] = S_{\ce{N2}}\) & \(\pdv{[\ce{N2}]}{x} = 0\)\\
        \bottomrule
    \end{tabular}
\end{table}

\subsection{Regulations}
\begin{align}
    DOM &< \SI{3}{\milli\gram\of{\ce{C}}\per\liter}\\
    \ce{NO3} &< \SI{25}{\milli\gram\of{\ce{NO3}}\per\liter}\\
    \ce{NH3} &< \SI{0.05}{\milli\gram\of{\ce{NH3}}\per\liter}
\end{align}

\section{Results}

\section{Discussion}
DOM infiltrates farther then expected because some removal processes are ignored.
\(k_1\) is the potential maximum removal rate which is never reached


\section{Conclusion}

\printbibliography
\end{document}